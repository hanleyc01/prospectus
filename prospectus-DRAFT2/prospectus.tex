\documentclass{article}
\usepackage{amsmath}
\usepackage{amsthm}
\usepackage{amssymb}
\usepackage{hyperref}
\usepackage[style=apa]{biblatex}
\usepackage[shortlabels]{enumitem}
\usepackage{graphicx}
\bibliography{sources.bib}

\newtheorem{theorem}{Theorem}[subsection]
\newtheorem{corollary}{Corollary}[subsection]
\newtheorem{lemma}{Lemma}[subsection]
\theoremstyle{definition}
\newtheorem{definition}{Definition}[subsection]

\title{Investigating Hebbian Alternatives to Dense Associative Memory}
\author{Connor Hanley}
\date{\today}

\begin{document}
\maketitle

\begin{abstract}
  Dense Associative Memories generalize traditional Hopfield networks
  while providing a substantially increased capacity. To do this,
  they show that by increasing the separation of similarity scores
  between query patterns and stored patterns. The cost of increased
  capacity is simplicity and biological plausibility. We propose
  to investigate the capacity characteristics of all Hebbian 
  associative memories, to see if there exists any alternative to 
  Dense Associative Memories which maintains the simplicity and 
  appeal of Hopfield networks.
\end{abstract}

Humans are able to recognize and retrieve patterns of data using distorted,
noisy, and partial patterns \parencite{rumelhart_general_1986}. This
capacity of human memory is known as \textit{content-addressability}: patterns
which are stored in memory are able to be ``looked up'' by themselves or their
parts. Modeling this property is a classical task in computational cognitive
and neuroscience (see \textcites{marr_simple_1971,little_existence_1974,
amari_learning_1972,nakano_associatron-model_1972,stanley_simulation_1976}).
The family of models which implement content-addressability are known
as \textit{associative memory models} (AMs).
A recent revival of interest AMs in machine learning research,
driven by their equivalence with ``attention'' layers in the transformer
architecture \parencites{vaswani_attention_2023, ramsauer_hopfield_2021},
has led to drastic advances in the storage capacity of AMs
\parencites{demircigil_model_2017,krotov_dense_2016,hu_provably_2024}.


\printbibliography
\end{document}